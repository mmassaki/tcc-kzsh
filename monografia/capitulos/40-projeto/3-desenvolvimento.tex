%
%~~
%~~ Desenvolvimento
%~~
%


\chapter{Desenvolvimento} \label{chp:desenvolvimento}

\section{Eventos} % (fold)
\label{sec:eventos}

\subsection{Temporiza��o do evento} % (fold)
\label{sub:temporiza��o_do_evento}

Os eventos informados pelos usu�rios s�o exibidos durante 30 minutos. Ap�s esse per�odo eles s�o marcados como inativos no banco de dados, n�o sendo mais informados pelo servidor.

Para a implementa��o da atualiza��o dos registros no banco de dados utilizou-se o ``crontab'' do servidor. O ``crontab'' � um utilit�rio dos sistemas Unix e Solaris que permite a execu��o de comandos (tarefas) em \emph{background} a cada intervalo de tempo especificado.

Criou-se ent�o, uma tarefa que atualiza o banco de dados inativando os eventos que ultrapassarem 30 minutos a partir do hor�rio em que foram informados. Essa tarefa � executada a cada cinco minutos no servidor para que n�o haja sobrecarga no mesmo.

A tarefa � descrita a seguir:
\lstset{language=Ruby}
\begin{lstlisting}
	require 'net::http'
	a = Evento.new
	b = a.list
	
\end{lstlisting}

% subsection temporiza��o_do_evento (end)

% section eventos (end)

\section{Registros} % (fold)
\label{sec:registros}

% section registros (end)

\section{Rota} % (fold)
\label{sec:rota}

% section rota (end)

\section{Tr�nsito} % (fold)
\label{sec:tr�nsito}

% section tr�nsito (end)

%%% Local Variables: 
%%% mode: latex
%%% TeX-master: "tese"
%%% x-symbol-8bits: t 
%%% End: 