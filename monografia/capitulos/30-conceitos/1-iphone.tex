%
%~~
%~~ iPhone
%~~
%

\chapter{iPhone} \label{chp:iphone}

Tendo sua primeira vers�o lan�ada em meados de 2007 pela Apple, o iPhone � um \emph{smartphone} repleto de recursos como c�mera de v�deo, microfone e alto-falantes, GPS, conectividade Wi-Fi e Bluetooth, aceler�metros, sensores de proximidade e luz ambiente e suporta os sinais GSM e 3G, dependendo do modelo do aparelho \cite{Apple:iPhoneSpecs:2010}. Seu grande apelo � a inovadora interface multi-toques que confere ao produto facilidade de uso e boa usabilidade.

Outro fator importante na popularidade do iPhone � a possibilidade de se instalar aplicativos. A Apple mant�m uma loja on-line, a App Store, onde � poss�vel se obter aplica��es, gratuitas e pagas, de diversas categorias, como jogos, ferramentas de produtividade, finan�as, not�cias, de redes sociais, navega��o, entre outras.

Essa grande diversidade de aplica��es, que j� ultrapassou a marca de 140.000, � desenvolvida e mantida por empresas e desenvolvedores de todo o mundo. O SDK (Software Development Kit - Kit de Desenvolvimento de Software) e as ferramentas de constru��o de interface e teste s�o gratuitas e podem ser encontradas no site de desenvolvimento da Apple \citep{Apple:DeveloperConnection:2010}. Existe muito material educacional para a plataforma e qualquer pessoa ou empresa pode obter a licen�a de desenvolvimento, mediante o pagamento de uma taxa.

Esse conjunto de caracter�sticas fez o iPhone atingir um alto n�vel de satisfa��o do usu�rio: 74\% deles dizem estar muito satisfeitos com seu aparelho \citep{ChangeWave:SmartPhoneMarket:2010}.

O resultado desse cen�rio � um grande e crescente n�mero de aparelhos vendidos. No quarto quadrimestre de 2009, a Apple vendeu 7,4 milh�es de unidades \citep{Apple:Reports:Q4:2009}, o que representa 7\% de crescimento em rela��o ao mesmo per�odo de 2008 e j� soma mais de 33 milh�es unidades vendidas. Esse n�mero representa 30\% do mercado de smartphones \citep{ChangeWave:SmartPhoneMarket:2010}, quantidade que parou de crescer com o lan�amento de uma plataforma concorrente - o Android - mas que ainda representa um mercado muito grande.

O iPhone usa GPS e triangula��o pelas antenas de celular e Wi-Fi para determinar sua posi��o. Os recursos de posicionamento s�o acess�veis a qualquer desenvolvedor via API.
%%% Local Variables: 
%%% mode: latex
%%% TeX-master: "tese"
%%% x-symbol-8bits: t 
%%% End: 