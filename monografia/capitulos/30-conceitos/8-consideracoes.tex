%
%~~
%~~ Considera��es finais
%~~
%
\section{Considera��es finais do cap�tulo} % (fold)
\label{sec:considera��es_finais_do_cap�tulo_2}

Este cap�tulo descreveu os principais conceitos te�ricos e tecnologias
envolvidas neste projeto de formatura. A compreens�o do conte�do dos itens anteriores facilita o entendimento do desenvolvimento do projeto.

O \emph{iPhone} foi selecionado como plataforma de desenvolvimento devido ao seu grande n�mero de usu�rios atualmente no Brasil, por fornecer ferramentas de desenvolvimento integradas e de uso simplificado e pelo interesse do grupo em aprofundar seus conhecimentos em desenvolvimento de aplicativos para \emph{iPhones}.

O uso do \emph{iPhone} como plataforma m�vel permitiu o uso do sistema A-GPS de posicionamento, dispon�vel nos aparelhos a partir da vers�o 3G, o uso do ambiente \emph{Xcode}, linguagem \emph{Objective-C} e \emph{Cocoa Touch}.

A implementa��o do servidor utilizando o \emph{framework} \emph{Ruby on Rails} permite um desenvolvimento mais f�cil e �gil, al�m de ser uma tecnologia de conhecimento do grupo e, juntamente com a ferramenta \emph{Scaffolding} permite a cria��o mais eficiente de CRUDs e interfaces \emph{web services} em padr�o \emph{RESTful}.

A \emph{Google Maps API} permite o acesso � informa��es geogr�ficas necess�rias para o mapeamento de tr�nsito e para a gera��o de rotas do sistema de forma gratuita e simples. Para o projeto adotamos a API via \emph{web services}, pois � a mais adequada para uso com o servidor, j� que usa somente requisi��es HTTP.

% section considera��es_finais_do_cap�tulo_2 (end)