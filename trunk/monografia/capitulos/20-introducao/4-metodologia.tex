%
%~~
%~~ Metodologia
%~~
%
\chapter{Metodologia de trabalho} \label{chp:metodologia}

A metodologia de trabalho utilizada do projeto de formatura seguiu as seguintes etapas:
\begin{enumerate}
	\item Estudo do estado da arte:
	
		A partir da id�ia de desenvolver uma aplica��o para coleta de informa��es de tr�nsito e disponibiliza��o dessas informa��es atualizadas para que as pessoas possam otimizar suas viagens dentro da cidade de S�o Paulo utilizando dispositivos m�veis, realizou-se um estudo do estado da arte de aplica��es do GPS em celulares.

	\item Escolha da tecnologia:
	
		Optou-se pelo dispositivo m�vel iPhone da Apple pela sua facilidade de desenvolvimento em rela��o a outras tecnologias, pelo interesse dos autores do projeto em aprofundar seus conhecimento nessa tecnologia e por j� atender os requisitos necess�rios para desenvolvimento nesta plataforma (celulares iPhones, computadores Mac e licen�a de desenvolvedor XCode).
		
		Para o servidor do sistema, optou-se pela tecnologia \emph{Ruby on Rails} pela facilidade e agilidade de implementa��o e pela sua grande refer�ncia em projetos de sistemas WEB e \emph{WebServices}.
		
	\item Defini��o de requisitos e escopo do sistema:
	
		Realizou-se reuni�es para defini��o das funcionalidades da aplica��o m�vel e seu escopo.	A partir das funcionalidades definidas levantou-se de requisitos do sistema a ser desenvolvido.
		
	\item Defini��o da arquitetura do sistema:
		
		
		
	\item Desenvolvimento e testes do sistema
		
		Nesta etapa desenvolveu-se as duas partes do sistema (servidor e aplicativo m�vel) com base nas especifica��es e arquiteturas definidas. Juntamente com o desenvolvimento efetuou-se testes unit�rios e testes de integra��o das duas partes dos sistema.
		
	\item Conclus�es:
		
		
		
\end{enumerate}



%%% Local Variables: 
%%% mode: latex
%%% TeX-master: "tese"
%%% x-symbol-8bits: t 
%%% End: 