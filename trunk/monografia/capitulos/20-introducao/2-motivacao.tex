%
% Motiva��o
%
% O que motivou o trabalho
% 
\chapter{Motiva��o} \label{chp:motivacao}

Tendo em vista a grande quantidade de carros que circulam atualmente pela cidade de S�o Paulo que v�m aumentando a cada ano - de Setembro de 2009 a Setembro de 2010 houve um aumento de 267.145 ve�culos saltando de 6.645.321 para 6.912.466 unidades \cite{Detran-SP:FrotaDeVeiculos} - � f�cil perceber que a quantidade de pistas pavimentadas e as obras de constru��es de novas pistas � ainda insuficiente gerando grandes congestionamentos por toda a cidade. 

Moradores da cidade de S�o Paulo e da Grande S�o Paulo enfrentam diariamente grandes congestionamentos durante seu trajeto de ida e volta seja para trabalho, lazer ou viagem perdendo muito tempo dentro do carro. A inten��o de reduzir e conseq�entemente aproveitar melhor este tempo perdido serviram de motiva��o para esse trabalho.


%%% Local Variables: 
%%% mode: latex
%%% TeX-master: "tese"
%%% x-symbol-8bits: t 
%%% End: