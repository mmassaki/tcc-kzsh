%
%~~
%~~ Considera��es finais
%~~
%
\section{Considera��es finais do cap�tulo} % (fold)
\label{sec:considera��es_finais_do_cap�tulo_3}

O desenvolvimento do sistema se mostrou um processo trabalhoso em alguns aspectos. Como a implementa��o das duas partes do sistema, servidor e aplicativo m�vel, foi realizada paralelamente as dificuldade puderam ser contornadas para que o sistema fosse conclu�do.

O uso do \emph{framework Ruby on Rails} permitiu uma implementa��o mais eficiente do servidor e que se focasse nas particularidades da implementa��o como o controle dos eventos, cria��o de rotas e mapeamento do tr�nsito.

O mapeamento do tr�nsito de forma colaborativa necessita de um n�mero m�nimo de usu�rio do sistema, pois n�o � poss�vel obter informa��es precisas apenas de um �nico motorista. E quanto maior a quantidade de usu�rio, maior a quantidade de ruas mapeadas e mais correta a informa��o ser�.

As APIs do \emph{Google Maps} permitiram em grande parte que as funcionalidades fossem implementadas de forma simples e sem muitas complica��es. No entanto h� a limita��o do servi�o em fornecer apenas um n�mero limitado de requisi��es por dia, limitando o uso do sistema.


% section considera��es_finais_do_cap�tulo_3 (end)