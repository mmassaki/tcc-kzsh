
% Acrônimos (siglas)
% uso: \acro{sigla}{descrição} cria um acrônimo e imprime ele num ambiente \begin{acronym}
%      \acrodef{sigla}{descrição} cria um acrônimo mas não imprime ele
%      \acf{sigla} escreve a por extenso e depois a (sigla)
%      \acl{sigla} escreve a por extenso
%      \ac{sigla} coloca a descriãoo da sigla se for sua primeira aparição

\pretextualchapter{Lista de Abreviaturas}% e Siglas}
\begin{acronym}
\acro{W3C}{World Wide Web Consortium}
\acro{XML}{Extensible Markup Language}
\acro{DTD}{Document Type Definition}
\acro{RDF}{Resource Description Framework}
\acro{RDFS}{RDF Schema}
\acro{OWL}{Web Ontology Language}
\acro{URI}{Uniform Resource Identifier}
\acro{SPARQL}{Simple Protocol and RDF Query Language}
\end{acronym}
 
%%% Local Variables: 
%%% mode: latex
%%% TeX-master: "tese"
%%% End: 
