
% Acr�nimos (siglas)
% uso: \acro{sigla}{descri��o} cria um acr�nimo e imprime ele num ambiente \begin{acronym}
%      \acrodef{sigla}{descri��o} cria um acr�nimo mas n�o imprime ele
%      \acf{sigla} escreve a por extenso e depois a (sigla)
%      \acl{sigla} escreve a por extenso
%      \ac{sigla} coloca a descri�oo da sigla se for sua primeira apari��o

\pretextualchapter{Lista de Abreviaturas e Siglas}
\begin{acronym}
	\acro{A-GPS}{\emph{Assisted global positioning system}}
	\acro{API}{\emph{Application programming interface}}
	\acro{CET}{Companhia de engenharia de tr�fego}
	\acro{CRUD}{\emph{Create, read, update and delete}}
	\acro{GPS}{\emph{Global positioning system}}
	\acro{HTTP}{\emph{Hypertext transfer protocol}}
	\acro{JSON}{\emph{JavaScript object notation}}
	\acro{TTFF}{\emph{Time to first fix}}
	\acro{UHF}{\emph{Ultra high frequency}}
	\acro{XML}{\emph{Extensible markup language}}
\end{acronym}
 
%%% Local Variables: 
%%% mode: latex
%%% TeX-master: "tese"
%%% End: 
